\documentclass[a4paper,justified]{tufte-book}
% Babel language package
% \usepackage[english,greek]{babel}
% \usepackage[utf8]{inputenc}
%%%%%%%%%%%%%%%%%%%%%%%%%%%%%%%%%%%%%%

% \usepackage{xltxtra} 
% \usepackage{xgreek} 
% \setmainfont[Mapping=tex-text]{GFS Didot} 

%%%%%%%%%%%%%%%%%%%%%%%%%%%%%%%%%%%%%%%
%\usepackage{kmath,kerkis} % The order of the packages matters  
%\usepackage[T1]{fontenc}


%%%%%%%%%%%%%%%%%%%%
\usepackage{fontspec}
\usepackage{libertine}
%%%%%%%%%%%%%%%%%%%%

\usepackage{amssymb, amsbsy}
\usepackage{amsmath, amsthm}
\usepackage{amsfonts}
\usepackage{mathtools}
\usepackage{empheq}
\usepackage{latexsym}
\usepackage{mathrsfs}
\usepackage{bm}
\usepackage{paralist}[2013/06/09]
\usepackage{fancyhdr}
\usepackage{epic, eepic}
\usepackage{geometry}
% \geometry{top=1in, bottom=2.5cm,left=1.cm,right=7.6cm,marginparwidth=5.5cm}
%%%%%%%%%%%%%%%%%%%%

\usepackage{multicol}
%%%%% math packages %%%%%%%%%%%%%%%%%%
\usepackage{proof}

\usepackage[italicdiff]{physics}
\usepackage[detect-all]{siunitx}
\usepackage{xfrac}
\usepackage[b]{esvect} 

%%%%%%% symbols packages %%%%%%%%%%%%%%
\usepackage{bm} %for use \bm instead \boldsymbol in math mode
%%%%%%%%%%%%%%%%%%%%%%%%%%%%%%%%%%%%%%%


%%%%%% graphics %%%%%%%%%%%%%%%%%%%%%%%
\usepackage{graphicx}
\usepackage{xcolor}
%\usepackage{xypic}
%\usepackage[all]{xy}
%\usepackage{calc}

%%%%%% tables %%%%%%%%%%%%%%%%%%%%%%%%%
\usepackage{array}
\usepackage{booktabs}
\usepackage{multirow}
\usepackage{makecell}
\usepackage{minibox}
\usepackage{systeme}
\usepackage{cite} 
%%%%%%%%%%%%%%%%%%%%%%%%%%%%%%%%%%%%%%%

\usepackage{enumerate}

\usepackage{fancyhdr}
%%%%% header and footer rule %%%%%%%%%
\setlength{\headheight}{14pt}
\renewcommand{\headrulewidth}{0pt}
\renewcommand{\footrulewidth}{0pt}
\fancypagestyle{plain}{\fancyhf{}
\fancyhead{}
\lfoot{\small \hrule \vspace{5pt}\color{magenta} Βαγγέλης Σαπουνάκης}
\cfoot{\small \hrule \vspace{5pt}\color{blue!75} Φοιτητικό Πρόσημο}
\rfoot{\small \hrule \vspace{5pt} \thepage}}
\fancypagestyle{vangelis}{\fancyhf{}
\rhead{\small }
\lhead{\small }
\lfoot{\small \hrule \vspace{5pt}\color{magenta} Βαγγέλης Σαπουνάκης}
\cfoot{\small \hrule \vspace{5pt}\color{blue!75} Φοιτητικό Πρόσημο}
\rfoot{\small \hrule \vspace{5pt} \thepage}}
%%%%%%%%%%%%%%%%%%%%%%%%%%%%%%%%%%%%%%%

%\usepackage{hyperref}
%\usepackage{url}
%%%%%%%% hyperref settings %%%%%%%%%%%%
%\hypersetup{pdfpagemode=UseOutlines,hidelinks,
%bookmarksopen=true,
%pdfdisplaydoctitle=true,
%pdfstartview=Fit,
%unicode=true,
%pdfpagelayout=OneColumn,
%}
%%%%%%%%%%%%%%%%%%%%%%%%%%%%%%%%%%%%%%%


% \usepackage{geometry}
% \geometry{left=25.63mm,right=25.63mm,top=36.25mm,bottom=36.25mm,footskip=24.16mm,headsep=24.16mm}

\usepackage{titlesec}
%%%%%% titlesec settings %%%%%%%%%%%%%
%\titleformat{\chapter}[block]{\LARGE\sc\bfseries}{\thechapter.}{1ex}{#1}
%\titlespacing*{\chapter}{0cm}{0cm}{36pt}[0ex]
%\titleformat{\section}[block]{\Large\bfseries}{\thesection.}{1ex}{#1}
%\titlespacing*{\section}{0cm}{34.56pt}{17.28pt}[0ex]
%\titleformat{\subsection}[block]{\large\bfseries{\thesubsection.}{1ex}{#1}
%\titlespacing*{\subsection}{0pt}{28.80pt}{14.40pt}[0ex]
%%%%%%%%%%%%%%%%%%%%%%%%%%%%%%%%%%%%%%

%%%%%%%%% My Theorems %%%%%%%%%%%%%%%%%%
\newtheorem{thm}{Θεώρημα}[]
\newtheorem{cor}[thm]{Πόρισμα}
\newtheorem{lem}[thm]{Λήμμα}
\newtheorem{prop}[thm]{Πρόταση}
\theoremstyle{definition}
\newtheorem{dfn}{Ορισμός}[]
\newtheorem{dfns}[dfn]{Ορισμοί}
\newtheorem{exam}[thm]{Παράδειγμα}
\newtheorem{exams}[thm]{Παραδείγματα}
\theoremstyle{remark}
\newtheorem{rem}{Παρατήρηση}[]
\newtheorem{rems}[rem]{Παρατηρήσεις}
%%%%%%%%%%%%%%%%%%%%%%%%%%%%%%%%%%%%%%%
\newtheoremstyle{break}
{\topsep}{\topsep}%
{\itshape}{}%
{\bfseries}{}%
{\newline}{}%

\theoremstyle{break}
\newtheorem{thmbreak}{Θεώρημα}
%%%%%%%%%%%%%%%%%%%%%%%%%%%%%%%%%%%%%%%

\usepackage[printwatermark]{xwatermark} 

\newwatermark[allpages,color=blue!3,angle=45,scale=3,xpos=0,ypos=0]{ΠΡΟΣΗΜΟ}

%%%%%%%%%%%%%%%%%%%%%%%%%%%%%%%%%%%%%%
%%%%%defines \inlineequation[<label name>]{<equation>}
%%%%% label inline equations and allow reference
\makeatletter
\newcommand*{\inlineequation}[2][]{%
	\begingroup
    % Put \refstepcounter at the beginning, because
	% package `hyperref' sets the anchor here.
	 \refstepcounter{equation}%
	\ifx\\#1\\%
\else
	 \label{#1}%
 \fi
 % prevent line breaks inside equation
 \relpenalty=10000 %
 \binoppenalty=10000 %
 \ensuremath{%
	 % \displaystyle % larger fractions, ...
	 #2%
 }%
 ~\@eqnnum
 \endgroup
}
\makeatother
%%%%%%%%%%%%%%%%%%%%%%%%%%%%%%%%%%%%%%%%%%%%%%%%%%%%

%%%\mybrace{<first>}{<second>}[<Optional text>]
%%% wrap with braces list environments
\NewDocumentCommand\mybrace{mmo}{%
\IfValueTF {#3}{%
\begin{tikzpicture}[overlay, remember picture,decoration={brace,amplitude=1ex}]
	ack] (#1.north east) -- (#2.south east) node[midway, right=0.1cm]
	{$\Rightarrow$}node[midway, right=0.5cm,text=black,text width = 2in,] {{#3}};
	\end{tikzpicture}%
}%
{%
	\begin{tikzpicture}[overlay, remember
		picture,decoration={brace,amplitude=1ex}]
		ack] (#1.north east) -- (#2.south east);
		\end{tikzpicture}%
	}%
}%a
%%%%%%%How to use this %%%%%%%%%%%%%%%%%%%%%%%%%
%use \tikzmark{a} and \tikzmark{b} at first and last \item where the brace is
%wanted
%use the following command after \end{enumerate}
%\mybrace{a}{b}[Text comes here to describe these to items and justify for your
%case]]


% \def\chpcolor{blue!45}
% \def\chpcolortxt{blue!60}
% \def\sectionfont{\sffamily\LARGE}

% \setcounter{secnumdepth}{2}


%%%%%%%%%%%%%%%%% My Sections %%%%%%%%%%%%%%%%%%%%%%%%%%%%%%%%%%
%\makeatletter
%%Section:
%\def\@sectionstrut{\vrule\@width\z@\@height12.5\p@}
%\def\@makesectionhead#1{%
%    {\par\vspace{20pt}%
%        \parindent 0pt\raggedleft\sectionfont
%        \colorbox{\chpcolor}{%
%            \parbox[t]{30pt}{\color{white}\@sectionstrut\@depth4.5\p@\hfill
% \ifnum\c@secnumdepth>\z@\thesection\fi}%
% }%
% \begin{minipage}[t]{\dimexpr\textwidth-10pt-2\fboxsep\relax}
%\color{\chpcolortxt}\@sectionstrut\hspace{5pt}#1
% \end{minipage}\par
%\vspace{10pt}%
%}
%}

% \def\section{\@afterindentfalse\secdef\@section\@ssection}
% \def\@section[#1]#2{%
%     \ifnum\c@secnumdepth>\m@ne
%         \refstepcounter{section}%
%         \addcontentsline{toc}{section}{\protect\numberline{\thesection}#1}%
%     \else
%         \phantomsection
%         \addcontentsline{toc}{section}{#1}%
%     \fi
%     \sectionmark{#1}%
% \if@twocolumn
%     \@topnewpage[\@makesectionhead{#2}]%
%  \else
%      \@makesectionhead{#2}\@afterheading
%  \fi
% }
% \def\@ssection#1{%A
% \if@twocolumn
%     \@topnewpage[\@makesectionhead{#1}]%
% \else
%     \@makesectionhead{#1}\@afterheading
% \fi
% }
% \makeatother

\setcounter{secnumdepth}{2}

 %chapter format
\titleformat{\chapter}[display]{\bfseries\Huge}{%
\begin{tikzpicture}
    \node[draw,rectangle,fill=blue!55,text=white]{\fontsize{40}{50}\selectfont\thechapter} ;
\end{tikzpicture}}
{0.5\baselineskip}{}[{\color{blue!55}\titlerule[2pt]}]

% section format
\titleformat{\section}%
  {\normalfont\LARGE\bfseries\itshape\color{blue!55}}% format applied to label+text
  {\llap{\colorbox{blue!55}{\parbox{0.9cm}{\hfill\color{white}\thesection}}}}% label
  {1em}% horizontal separation between label and title body
  {}% before the title body
  []% after the title body

% subsection format
\titleformat{\subsection}%
  {\normalfont\Large\bfseries\itshape\color{blue!55}}% format applied to label+text
  {\llap{\colorbox{blue!55}{\parbox{0.9cm}{\hfill\color{white}\thesubsection}}}}% label
  {1em}% horizontal separation between label and title body
  {}% before the title body
  []% after the title body 

