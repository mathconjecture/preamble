\documentclass[a4paper,table,12pt]{report}
\usepackage{etex}

\usepackage{ifxetex}
\ifxetex
  % IF XELATEX
  \usepackage{fontspec}
  % \usepackage{libertine}
  \usepackage{polyglossia}
  \setdefaultlanguage{greek}
  \setotherlanguage{english}
  \setmainfont[Extension=.ttf,UprightFont=*-Regular,BoldFont=*-Bold,ItalicFont=*-Italic,BoldItalicFont=*-Bold-Italic]{Minion-Pro}
  \setsansfont[Extension=.ttf,UprightFont=*H,BoldFont=*HB,ItalicFont=*HI,BoldItalicFont=*HBI]{Vera}

\else
  % IF PDFLATEX
  \usepackage[T1]{fontenc}
  \usepackage[utf8]{inputenc}
  \usepackage[english,greek]{babel}
\fi



\usepackage[shrink=10,stretch=10]{microtype} 
%%%%% math packages %%%%%%%%%%%%%%%%%%
\usepackage[intlimits]{amsmath}
\usepackage{amssymb}
\usepackage{amsfonts}
\usepackage{amsthm}
\usepackage{proof}
\usepackage{mathtools}
\usepackage[makeroom]{cancel}
\usepackage{xspace}
\usepackage{subfiles}
\usepackage{xr}
\usepackage{mathrsfs} %for Laplace

\usepackage[italicdiff]{physics}
\usepackage[detect-all]{siunitx}
\usepackage{xfrac}
\usepackage[b]{esvect} 

%%%%%%% symbols packages %%%%%%%%%%%%%%
\usepackage{bm} %for use \bm instead \boldsymbol in math mode
% \usepackage{dsfont}
% \usepackage{stmaryrd}

\usepackage{anyfontsize}
\newlength{\FONTmain}\setlength{\FONTmain}{10pt}
\newlength{\FONTmainbl}\setlength{\FONTmainbl}{1.2\FONTmain}
\renewcommand{\tiny}        {\fontsize{0.6\FONTmain}{0.6\FONTmainbl}\selectfont}
\renewcommand{\scriptsize}  {\fontsize{0.7\FONTmain}{0.7\FONTmainbl}\selectfont}
\renewcommand{\footnotesize}{\fontsize{0.8\FONTmain}{0.8\FONTmainbl}\selectfont}
\renewcommand{\small}       {\fontsize{0.9\FONTmain}{0.9\FONTmainbl}\selectfont}
\renewcommand{\normalsize}  {\fontsize{1.0\FONTmain}{1.0\FONTmainbl}\selectfont}
\renewcommand{\large}       {\fontsize{1.2\FONTmain}{1.2\FONTmainbl}\selectfont}
\renewcommand{\Large}       {\fontsize{1.4\FONTmain}{1.4\FONTmainbl}\selectfont}
\renewcommand{\LARGE}       {\fontsize{1.6\FONTmain}{1.6\FONTmainbl}\selectfont}
\renewcommand{\huge}        {\fontsize{1.8\FONTmain}{1.8\FONTmainbl}\selectfont}

%%%%%%%%%%%%%%%%%%%%%%%%%%%%%%%%%%%%%%%



%%%%%%%%%%%% graphics %%%%%%%%%%%%%%%%%%%%%%%
\usepackage{graphicx}
\usepackage{multicol}

\usepackage[RGB]{xcolor}
\definecolor{Col1}{rgb}{0.74, 0.2, 0.64}
\definecolor{Col2}{rgb}{0.0, 0.55, 0.55}
\definecolor{Col3}{rgb}{0.74, 0.2, 0.64}
\definecolor{Col4}{rgb}{0.0, 0.55, 0.55}
\definecolor{Col5}{rgb}{0.74, 0.2, 0.64}


\usepackage[many]{tcolorbox}
%%%%%%%%%%%%% TIKZ $$$$$$$$$$$$$$$$$
\usepackage{tikz}
\usetikzlibrary{shapes,angles,calc,arrows,arrows.meta,quotes,intersections}
\usetikzlibrary{decorations.pathmorphing}
\usetikzlibrary{decorations.pathreplacing} 
\usetikzlibrary{decorations.markings,patterns}
\usetikzlibrary{positioning}
\usetikzlibrary{calc}
\usepackage{pgfplots}
\usepgfplotslibrary{fillbetween}

\pgfplotsset{compat=1.10}
\usepgfplotslibrary{fillbetween}

\tikzset{dot/.style={draw, fill, circle, inner sep=1pt, minimum size=3pt}}

%%%%%%%%%%%%%%%%%%% My Box Styles %%%%%%%%%%%%%%%%%%%%%%%%%
\tikzset{myboxdfn/.style = {draw=Col1!75,rectangle, ultra thick, fill=Col1!6, inner sep=10pt},
  myboxtitledfn/.style={rectangle, fill=Col1!75, text=white, font={\bfseries},right=2\baselineskip},
  myboxthm/.style = {draw=Col2!75,rectangle, ultra thick, fill=Col2!6, inner sep=10pt},
  myboxtitlethm/.style={rectangle, fill=Col2!75, text=white, font={\bfseries},right=2\baselineskip},
  myboxprop/.style = {draw=Col2!35,rectangle, ultra thick, fill=Col2!2, inner sep=10pt},
myboxtitleprop/.style={rectangle, fill=Col2!35, text=white, font={\bfseries},right=2\baselineskip}}



%\usepackage{xypic}
%\usepackage[all]{xy}
%\usepackage{calc}

%%%%%% tables %%%%%%%%%%%%%%%%%%%%%%%%%
\usepackage{array}
\usepackage{booktabs}
\usepackage{multirow}
\usepackage{makecell}
\usepackage{minibox}
\usepackage{systeme}
\usepackage{cite} 
\usepackage{extarrows} 


\usepackage[shortlabels,inline]{enumitem}

%\usepackage{fancyhdr}
%%%%%% header and footer rule %%%%%%%%%
%\setlength{\headheight}{14pt}
%\renewcommand{\headrulewidth}{0pt}
%\renewcommand{\footrulewidth}{0.5pt}
% \fancypagestyle{plain}
%{
%  \fancyhf{}
%  \fancyhead{}
%  \lfoot{\small \vspace{3pt}\color{Col1} Βαγγέλης Σαπουνάκης}
%  \cfoot{\small \vspace{3pt}\color{Col2!75} Φοιτητικό Πρόσημο}
%  \rfoot{\small \vspace{3pt} \thepage}
%}
%\fancypagestyle{vangelis}
%{
%  \fancyhf{}
%  \lfoot{\small \vspace{3pt}\color{Col1} Βαγγέλης Σαπουνάκης}
%  \cfoot{\small \vspace{3pt}\color{Col2!75} Φοιτητικό Πρόσημο}
%  \rfoot{\small \vspace{3pt} \thepage}
%}
%\fancypagestyle{askhseis}
%{
%  \fancyhf{}
%  \lfoot{\small \vspace{3pt}\color{Col1} Φοιτητικό Πρόσημο}
%  \cfoot{\small \vspace{3pt}\color{Col2!75} }
%  \rfoot{\small \vspace{3pt} \thepage}
%}
%%%%%%%%%%%%%%%%%%%%%%%%%%%%%%%%%%%%%%%%

\usepackage{hyperref}
\usepackage{url}
%%%%%%% hyperref settings %%%%%%%%%%%%
\hypersetup{pdfpagemode=UseOutlines,hidelinks,
  colorlinks=true,
  % linkcolor=Col1,
  % anchorcolor= ,
  % citecolor=Col1 ,
  % urlcolor= ,
  allcolors=Col1,
  bookmarksopen=true,
  pdfdisplaydoctitle=true,
  pdfstartview=Fit,
  unicode=true,
  pdfpagelayout=OneColumn,
}

%%%%%%%%%%%%%%%%%%%%%%%%%%%%%%%%%%%%%%


\usepackage[space]{grffile}


\usepackage{geometry}
\geometry{left=25.63mm,right=25.63mm,top=36.25mm,bottom=36.25mm,footskip=24.16mm,headsep=24.16mm}

\usepackage{titlesec}
%%%%% titlesec settings %%%%%%%%%%%%%
\titleformat{\chapter}[block]{\bfseries\Huge\color{Col1}}{\thechapter.}{1ex}{}
\titlespacing*{\chapter}{0cm}{0cm}{36pt}[0ex]
\titleformat{\section}[block]{\Large\bfseries\itshape\color{Col1!85}}{\thesection.}{1ex}{}
\titlespacing*{\section}{0pt}{\baselineskip}{0.5\baselineskip}[0ex]
\titleformat{\subsection}[block]{\large\bfseries\itshape\color{Col1!50}}{\thesubsection.}{1ex}{}
\titlespacing*{\subsection}{0pt}{28.80pt}{14.40pt}[0ex]
%%%%%%%%%%%%%%%%%%%%%%%%%%%%%%%%%%%%%


%%%%%%%%% My Theorems %%%%%%%%%%%%%%%%%%
\theoremstyle{definition}
\newtheorem{thm}{Θεώρημα}[section]
\newtheorem{cor}[thm]{Πόρισμα}
\newtheorem{lem}[thm]{Λήμμα}
\newtheorem{prop}[thm]{Πρόταση}
\newtheorem{isx}[thm]{Ισχυρισμός}
\theoremstyle{definition}
\newtheorem{dfn}{Ορισμός}[]
\newtheorem{dfns}[dfn]{Ορισμοί}
%%%%%%%%%%%%%%%%%%%%%%%%%%%%%%%%%%%%%%%

% \newtheoremstyle{note} %〈name〉
% {3pt} %〈Space above〉
% {3pt} %〈Space below〉
% {} %〈Body font〉
% {} %〈Indent amount〉
% {\itshape} %〈Theorem head font〉
% {:} %〈Punctuation after theorem head〉
% {.5em} %〈Space after theorem head〉
% {} %〈Theorem head spec(can be left empty, meaning ‘normal’)〉

\newtheoremstyle{break}
{\topsep}{\topsep}%
{\itshape}{}%
{\bfseries}{}%
{\newline}{}%

\theoremstyle{break}
\newtheorem{thmbreak}{Θεώρημα}

\newtheoremstyle{myrem}
{\topsep}{\topsep}%
{}{}%
{\color{Col2}\bfseries}{}%
{\baselineskip}{}%

\theoremstyle{myrem}
\newtheorem{rem}[thm]{Παρατήρηση}
\newtheorem{rems}[thm]{Παρατηρήσεις}

\newtheoremstyle{myex}
{\topsep}{\topsep}%
{}{}%
{\color{Col1}\bfseries}{}%
{\baselineskip}{}%

\theoremstyle{myex}
\newtheorem{example}[thm]{Παράδειγμα}
\newtheorem{examples}[thm]{Παραδείγματα}
\newtheorem{exercise}[thm]{Άσκηση}
\newtheorem{exercises}[thm]{Ασκήσεις}

\newenvironment{solution}[1][\bfseries Λύση]
{\begin{proof}[#1]}
{\end{proof}}


% \usepackage[printwatermark]{xwatermark} 
% \newwatermark[
% allpages,
% fontfamily=lmr,
% fontseries=m,
% color=Col2!30,
% angle=45,
% scale=4,
% xpos=0,
% ypos=0
% ]{ΠΡΟΣΗΜΟ}
%%%%%%%%%%%%%%%%%%%%%%%%%%%%%%%%%%%%%%%

%%%%%%%%%%%%%Watermark%%%%%%%%%%%%%%%%%%
% \usepackage{background}
% \backgroundsetup{angle=45,color=Col2!20,opacity=0.6,contents={\large{ΠΡΟΣΗΜΟ}}}
%%%%%%%%%%%%%%%%%%%%%%%%%%%%%%%%%%%%%%%

%%%%%%%%%%%%%%%% WATERMARK above pictures with opacity %%%%%%%%%%%%%%%%%%%%%%%%%%%%%%
%\usepackage[printwatermark]{xwatermark}
%\newsavebox\mybox
%\savebox\mybox{\tikz[color=Col2,opacity=0.2]\node{ΠΡΟΣΗΜΟ};}
%\newwatermark*[
%pages=2-\lastdocpage,
%fontfamily=lmr,
%fontseries=m,
%angle=45,
%scale=12,
%xpos=-40,
%ypos=40,
%]{\usebox\mybox} 
%%%%%%%%%%%%%%%%%%%%%%%%%%%%%%%%%%%%%%%%%%%%%%%%%%%%%%%


\setlength{\parindent}{\baselineskip}

\setlength{\columnseprule}{0pt}
\setlength{\columnsep}{1\baselineskip}
\renewcommand{\columnseprulecolor}{\color{Col1}}

