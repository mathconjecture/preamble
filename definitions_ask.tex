\DeclareMathOperator{\Arg}{Arg}
\DeclareMathOperator{\Span}{span}


% --- Macro \xvec
\makeatletter
\newlength\xvec@height%
\newlength\xvec@depth%
\newlength\xvec@width%
\newcommand{\xvec}[2][]{%
    \ifmmode%
        \settoheight{\xvec@height}{$#2$}%
        \settodepth{\xvec@depth}{$#2$}%
        \settowidth{\xvec@width}{$#2$}%
    \else%
        \settoheight{\xvec@height}{#2}%
        \settodepth{\xvec@depth}{#2}%
        \settowidth{\xvec@width}{#2}%
    \fi%
    \def\xvec@arg{#1}%
    \def\xvec@dd{:}%
    \def\xvec@d{.}%
    \raisebox{.2ex}{\raisebox{\xvec@height}{\rlap{%
                \kern.05em%  (Because left edge of drawing is at .05em)
                \begin{tikzpicture}[scale=1]
                    \pgfsetroundcap
                    \draw (.05em,0)--(\xvec@width-.05em,0);
                    \draw (\xvec@width-.05em,0)--(\xvec@width-.15em, .075em);
                    \draw (\xvec@width-.05em,0)--(\xvec@width-.15em,-.075em);
                    \ifx\xvec@arg\xvec@d%
                        \fill(\xvec@width*.45,.5ex) circle (.5pt);%
                    \else\ifx\xvec@arg\xvec@dd%
                        \fill(\xvec@width*.30,.5ex) circle (.5pt);%
                        \fill(\xvec@width*.65,.5ex) circle (.5pt);%
                    \fi\fi%
                \end{tikzpicture}%
    }}}%
    #2%
}
\makeatother

% --- Override \vec with an invocation of \xvec.
\let\stdvec\vec
\renewcommand{\vec}[1]{\xvec[]{#1}}
% --- Define \dvec and \ddvec for dotted and double-dotted vectors.
\newcommand{\dvec}[1]{\xvec[.]{#1}}
\newcommand{\ddvec}[1]{\xvec[:]{#1}}


%%%%%%%%%%%%%%%%%% fancy headings %%%%%%%%%%%%%%%%%%
\setcounter{secnumdepth}{2}

%chapter format
\titleformat{\chapter}[display]%
{\bfseries\Huge\color{Col1}}% format applied to label+text
{\begin{tikzpicture}
        \node[rectangle,fill=Col1,text=white]{\itshape\fontsize{40}{50}\selectfont
        \thechapter} ;
\end{tikzpicture}}%label
{1.5\baselineskip}% horizontal separation between label and title body
{}% before the title body
[]% after the title body

% section format
\titleformat{\section}%
{\normalfont\Large\bfseries\itshape\color{Col2!85}}% format applied to label+text
{\llap{\colorbox{Col2!85}{\parbox{0.65cm}{\hfill\color{white}\thesection}}}}% label
{1em}% horizontal separation between label and title body
{}% before the title body
[] % after the title body 
% subsection format
\titleformat{\subsection}%
{\normalfont\large\bfseries\itshape\color{Col2!50}}% format applied to label+text
{\llap{\colorbox{Col2!50}{\parbox{0.55cm}{\hfill\color{white}\thesubsection}}}}% label
{1em}% horizontal separation between label and title body
{}% before the title body
[]% after the title body 


%%%%%%%%%%%%%%%% Fancy Tables %%%%%%%%%%%%%%%%%%%%%%%%%%%%
\usepackage{calc}
\usepackage{array}
\definecolor{TabLine}{RGB}{254,254,254}
\newcommand{\TabRowHead}{\rowcolor{TabHeadRow}}
\newcommand{\TabRowHeadCor}{\cellcolor{white}}
\newcommand{\TabRowHCol}{\color{white}\bfseries\boldmath}
\newcommand{\TabCellHead}{\cellcolor{TabHeadRow}\TabRowHCol}
\newenvironment{Mytable}%
{\begingroup\setlength{\arrayrulewidth}{2pt}\arrayrulecolor{TabLine}
    \colorlet{TabHeadRow}{Col\thechapter}
    \colorlet{TabRowOdd}{Col\thechapter!50!white}
    \colorlet{TabRowEven}{Col\thechapter!25!white}
    \rowcolors{1}{TabRowOdd}{TabRowEven}
}%
{\endgroup

}


%%%%%%%%%%%%%%%%%%%%% 2 columns $$$$$$$$$$$$$$$$$$$$$$
\newcommand{\twocolumnside}[2]{\begin{minipage}[t]{0.45\linewidth}\raggedright
        #1
        \end{minipage}\hfill{\color{Col1}{\vrule width 1pt}}\hfill\begin{minipage}[t]{0.45\linewidth}\raggedright
        #2
    \end{minipage}
}

\newcommand{\twocolumnsides}[2]{\begin{minipage}[t]{0.45\linewidth}\raggedright
        #1
        \end{minipage}\hfill\begin{minipage}[t]{0.45\linewidth}\raggedright
        #2
    \end{minipage}
}

%%%%%%%%%%%%%%%%%%%%%%% my boxes %%%%%%%%%%%%%%%%%%%%%%%%%%%%
\newcommand{\mythm}[1]{
    \begin{tikzpicture}
        \node[myboxthm] (box1) 
        {
            \begin{minipage}{0.9\textwidth}
                #1
            \end{minipage}
        } ;

        \node[myboxtitlethm] at (box1.north
        west) {Θεώρημα} ;
    \end{tikzpicture}
}

\newcommand{\mydfn}[1]{
    \begin{tikzpicture}
        \node[myboxdfn] (box1) 
        {
            \begin{minipage}{0.9\textwidth}
                #1
            \end{minipage}
        } ;

        \node[myboxtitledfn] at (box1.north
        west) {Ορισμός} ;
    \end{tikzpicture}
}


\newcommand{\myprop}[1]{
    \begin{tikzpicture}
        \node[myboxprop] (box1) 
        {
            \begin{minipage}{0.9\textwidth}
                #1
            \end{minipage}
        } ;

        \node[myboxtitleprop] at (box1.north
        west) {Πρόταση} ;
    \end{tikzpicture}
}

%%%\mybrace{<first>}{<second>}[<Optional text>]
\newcommand{\tikzmark}[1]{\tikz[baseline={(#1.base)},overlay,remember picture] \node[outer sep=0pt, inner sep=0pt] (#1) {\phantom{A}};}
%% syntax
\NewDocumentCommand\mybrace{mmo}{%
    \IfValueTF {#3}{%
        \begin{tikzpicture}[overlay, remember picture,decoration={brace,amplitude=1ex}]
            \draw[decorate,thick] (#1.north east) -- (#2.south east) node[midway, right=0.1cm] 
            {$\Rightarrow$}node[midway, right=0.5cm,text=black,text width = 2in,] {{#3}};
        \end{tikzpicture}%
    }%
    {%
        \begin{tikzpicture}[overlay, remember picture,decoration={brace,amplitude=1ex}]
            \draw[decorate,thick] (#1.north east) -- (#2.south east);
        \end{tikzpicture}%
    }%
}%

%%%%%%%How to use this %%%%%%%%%%%%%%%%%%%%%%%%%
%use \tikzmark{a} and \tikzmark{b} at first and last \item where the brace is
%wanted
%use the following command after \end{enumerate}
%\mybrace{a}{b}[Text comes here to describe these to items and justify for your
%case]]



%%%%% label inline equations and don't allow reference
\newcommand\inlineeqno{\stepcounter{equation}\quad (\theequation)}

%%%%%defines \inlineequation[<label name>]{<equation>}
%%%%%%%%format use \inlineequation[<label name>]{<equation>}%%%%%%%
\makeatletter
\newcommand*{\inlineequation}[2][]{%
    \begingroup
    % Put \refstepcounter at the beginning, because
    % package `hyperref' sets the anchor here.
    \refstepcounter{equation}%
    \ifx\\#1\\%
\else
    \label{#1}%
\fi
% prevent line breaks inside equation
\relpenalty=10000 %
\binoppenalty=10000 %
\ensuremath{%
    % \displaystyle % larger fractions, ...
    #2%
}%
\, ~\@eqnnum
\endgroup
}
\makeatother


%%%%%%%%%%%%%%%%%% fancy enumitem cicled label %%%%%%%%%%%%%%%%%%
\newcommand*\circled[1]{\tikz[baseline=(char.base)]{
\node[shape=circle,draw,inner sep=0.3pt] (char) {#1};}}
% use it like \begin{enumerate}[label=\protect\circled{\Alph{enumi}}]
%%%\mybrace{<first>}{<second>}[<Optional text>]
%%% wrap with braces list environments


\newlist{myitemize}{itemize}{3}
\setlist[myitemize]{label=\textcolor{Col1}{\tiny$\blacksquare$}}


%%%%%%%%%%%%% puts brace under matrix
\newcommand\undermat[2]{%
  \makebox[0pt][l]{$\smash{\underbrace{\phantom{%
\begin{matrix}#2\end{matrix}}}_{\text{$#1$}}}$}#2}


%circle item inside array or matrix
\newcommand\Circle[1]{%
\tikz[baseline=(char.base)]\node[circle,draw,inner sep=2pt] (char) {#1};}

