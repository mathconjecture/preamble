\documentclass[a4paper]{report}
% \usepackage{etex}
%%% Babel language package
%\usepackage[english,greek]{babel}
%%% Inputenc font encoding
%\usepackage[utf8]{inputenc}

\usepackage{fontspec}
\usepackage{libertine}
\usepackage{polyglossia}
\setdefaultlanguage{greek}
\setotherlanguage{english}

\usepackage[shrink=10,stretch=10]{microtype} 
%%%%% math packages %%%%%%%%%%%%%%%%%%
\usepackage[intlimits]{amsmath}
\usepackage{amssymb}
\usepackage{amsfonts}
\usepackage{amsthm}
\usepackage{proof}
\usepackage{mathtools}
\usepackage{xspace}

\usepackage[italicdiff]{physics}
\usepackage[detect-all]{siunitx}
\usepackage{xfrac}
\usepackage[b]{esvect} 

%%%%%%% symbols packages %%%%%%%%%%%%%%
\usepackage{bbm} %for use \bm instead \boldsymbol in math mode
% \usepackage{dsfont}
% \usepackage{stmaryrd}
\usepackage{anyfontsize}

\newlength{\FONTmain}\setlength{\FONTmain}{12pt}
\newlength{\FONTmainbl}\setlength{\FONTmainbl}{1.2\FONTmain}
\renewcommand{\tiny}        {\fontsize{0.6\FONTmain}{0.6\FONTmainbl}\selectfont}
\renewcommand{\scriptsize}  {\fontsize{0.7\FONTmain}{0.7\FONTmainbl}\selectfont}
\renewcommand{\footnotesize}{\fontsize{0.8\FONTmain}{0.8\FONTmainbl}\selectfont}
\renewcommand{\small}       {\fontsize{0.9\FONTmain}{0.9\FONTmainbl}\selectfont}
\renewcommand{\normalsize}  {\fontsize{1.0\FONTmain}{1.0\FONTmainbl}\selectfont}
\renewcommand{\large}       {\fontsize{1.2\FONTmain}{1.2\FONTmainbl}\selectfont}
\renewcommand{\Large}       {\fontsize{1.4\FONTmain}{1.4\FONTmainbl}\selectfont}
\renewcommand{\LARGE}       {\fontsize{1.6\FONTmain}{1.6\FONTmainbl}\selectfont}
\renewcommand{\huge}        {\fontsize{1.8\FONTmain}{1.8\FONTmainbl}\selectfont}
%%%%%%%%%%%%%%%%%%%%%%%%%%%%%%%%%%%%%%%



%%%%%%%%%%%% graphics %%%%%%%%%%%%%%%%%%%%%%%
\usepackage{graphicx}
\usepackage[most]{tcolorbox}
\usepackage{multicol}

\usepackage[dvipsnames,table,RGB]{xcolor}
\definecolor{Col1}{rgb}{0.74, 0.2, 0.64}
\definecolor{Col2}{rgb}{0.0, 0.55, 0.55}
\definecolor{Col3}{rgb}{0.74, 0.2, 0.64}
\definecolor{Col4}{rgb}{0.0, 0.55, 0.55}
\definecolor{Col5}{rgb}{0.74, 0.2, 0.64}
\definecolor{Col6}{rgb}{0.0, 0.55, 0.55}
\definecolor{Col7}{rgb}{0.74, 0.2, 0.64}
\definecolor{Col8}{rgb}{0.0, 0.55, 0.55}

\definecolor{cerulean}{rgb}{0.0, 0.48, 0.65}
\definecolor{coolgrey}{rgb}{0.55, 0.57, 0.67}
\definecolor{darkcyan}{rgb}{0.0, 0.55, 0.55}
\definecolor{darklavender}{rgb}{0.45, 0.31, 0.59}
\definecolor{darkpastelblue}{rgb}{0.47, 0.62, 0.8}
\definecolor{lavenderblue}{rgb}{0.8, 0.8, 1.0}

%%%%%%%%%%%%% TIKZ $$$$$$$$$$$$$$$$$
\usepackage{tikz}
\usetikzlibrary{shapes,angles,calc,arrows,arrows.meta,quotes,intersections}
\usetikzlibrary{decorations.pathmorphing}
\usetikzlibrary{decorations.pathreplacing} 
\usetikzlibrary{decorations.markings,patterns}
\usepackage{pgfplots}

\pgfplotsset{compat=1.10}
\usepgfplotslibrary{fillbetween}

\tikzset{dot/.style={ draw, fill, circle, inner sep=1pt, minimum size=3pt }}

%%%%%%%%%%%%%%%%%%% My Box Styles %%%%%%%%%%%%%%%%%%%%%%%%%
\tikzset{myboxdfn/.style = {draw=lavenderblue!55,rectangle, ultra thick, fill=lavenderblue!6, inner sep=10pt},
myboxtitledfn/.style={rectangle, fill=lavenderblue!55, text=white, font={\bfseries},right=2\baselineskip},
myboxthm/.style = {draw=magenta!55,rectangle, ultra thick, fill=magenta!6, inner sep=10pt},
myboxtitlethm/.style={rectangle, fill=magenta!55, text=white, font={\bfseries},right=2\baselineskip}}



\usepackage{fancyhdr}
%\usepackage{xypic}
%\usepackage[all]{xy}
%\usepackage{calc}

%%%%%% tables %%%%%%%%%%%%%%%%%%%%%%%%%
\usepackage{array}
\usepackage{booktabs}
\usepackage{multirow}
\usepackage{makecell}
\usepackage{minibox}
\usepackage{systeme}
\usepackage{cite} 
\usepackage{extarrows} 


\usepackage[shortlabels,inline]{enumitem}

\usepackage{fancyhdr}
%%%%% header and footer rule %%%%%%%%%
\setlength{\headheight}{14pt}
\renewcommand{\headrulewidth}{0pt}
\renewcommand{\footrulewidth}{0pt}
\fancypagestyle{plain}{\fancyhf{}
\fancyhead{}
\lfoot{\small \hrule \vspace{5pt}\color{magenta} Βαγγέλης Σαπουνάκης}
\cfoot{\small \hrule \vspace{5pt}\color{blue!75} Φοιτητικό Πρόσημο}
\rfoot{\small \hrule \vspace{5pt} \thepage}}
\fancypagestyle{vangelis}{\fancyhf{}
\rhead{\small }
\lhead{\small }
\lfoot{\small \hrule \vspace{5pt}\color{magenta} Βαγγέλης Σαπουνάκης}
\cfoot{\small \hrule \vspace{5pt}\color{blue!75} Φοιτητικό Πρόσημο}
\rfoot{\small \hrule \vspace{5pt} \thepage}}
%%%%%%%%%%%%%%%%%%%%%%%%%%%%%%%%%%%%%%%
%\usepackage{hyperref}
%\usepackage{url}
%%%%%%%% hyperref settings %%%%%%%%%%%%
%\hypersetup{pdfpagemode=UseOutlines,hidelinks,
%bookmarksopen=true,
%pdfdisplaydoctitle=true,
%pdfstartview=Fit,
%unicode=true,
%pdfpagelayout=OneColumn,
%}
%%%%%%%%%%%%%%%%%%%%%%%%%%%%%%%%%%%%%%%

\usepackage[space]{grffile}

\usepackage{geometry}
\geometry{left=25.63mm,right=25.63mm,top=36.25mm,bottom=36.25mm,footskip=24.16mm,headsep=24.16mm}

\usepackage{titlesec}
%%%%%% titlesec settings %%%%%%%%%%%%%
%\titleformat{\chapter}[block]{\LARGE\sc\bfseries}{\thechapter.}{1ex}{#1}
%\titlespacing*{\chapter}{0cm}{0cm}{36pt}[0ex]
%\titleformat{\section}[block]{\Large\bfseries}{\thesection.}{1ex}{#1}
%\titlespacing*{\section}{0cm}{34.56pt}{17.28pt}[0ex]
%\titleformat{\subsection}[block]{\large\bfseries{\thesubsection.}{1ex}{#1}
%\titlespacing*{\subsection}{0pt}{28.80pt}{14.40pt}[0ex]
%%%%%%%%%%%%%%%%%%%%%%%%%%%%%%%%%%%%%%

%%%%%%%%% My Theorems %%%%%%%%%%%%%%%%%%
\newtheorem{thm}{Θεώρημα}[]
\newtheorem{cor}[thm]{Πόρισμα}
\newtheorem{lem}[thm]{Λήμμα}
\newtheorem{prop}[thm]{Πρόταση}
\newtheorem{isx}[thm]{Ισχυρισμός}
\theoremstyle{definition}
\newtheorem{dfn}{Ορισμός}[]
\newtheorem{dfns}[dfn]{Ορισμοί}
\newtheorem{exam}[thm]{Παράδειγμα}
\newtheorem{exams}[thm]{Παραδείγματα}
\theoremstyle{remark}
\newtheorem{rem}{Παρατήρηση}[]
\newtheorem{rems}[rem]{Παρατηρήσεις}
%%%%%%%%%%%%%%%%%%%%%%%%%%%%%%%%%%%%%%%
\newtheoremstyle{break}
{\topsep}{\topsep}%
{\itshape}{}%
{\bfseries}{}%
{\newline}{}%

\theoremstyle{break}
\newtheorem{thmbreak}{Θεώρημα}
%%%%%%%%%%%%%%%%%%%%%%%%%%%%%%%%%%%%%%%

%%%%%%%%%%%%%Watermark%%%%%%%%%%%%%%%%%%
% \usepackage[printwatermark]{xwatermark} 
% \newwatermark[allpages,color=blue!8,angle=45,scale=3,xpos=0,ypos=0]{ΠΡΟΣΗΜΟ}
%%%%%%%%%%%%%%%%%%%%%%%%%%%%%%%%%%%%%%%

 %%%%%%%%%%%%%%%%%%% fancy enumitem cicled label %%%%%%%%%%%%%%%%%%
\newcommand*\circled[1]{\tikz[baseline=(char.base)]{
            \node[shape=circle,draw,inner sep=0.3pt] (char) {#1};}}
% use it like \begin{enumerate}[label=\protect\circled{\Alph{enumi}}]

%%%%%%%%%%%%%%%%%%%%%%%%%%%%%%%%%%%%%%
%%%%%defines \inlineequation[<label name>]{<equation>}
%%%%% label inline equations and allow reference
\makeatletter
\newcommand*{\inlineequation}[2][]{%
	\begingroup
    % Put \refstepcounter at the beginning, because
	% package `hyperref' sets the anchor here.
	 \refstepcounter{equation}%
	\ifx\\#1\\%
\else
	 \label{#1}%
 \fi
 % prevent line breaks inside equation
 \relpenalty=10000 %
 \binoppenalty=10000 %
 \ensuremath{%
	 % \displaystyle % larger fractions, ...
	 #2%
 }%
 ~\@eqnnum
 \endgroup
}
\makeatother
%%%%%%%%%%%%%%%%%%%%%%%%%%%%%%%%%%%%%%%%%%%%%%%%%%%%

%%%\mybrace{<first>}{<second>}[<Optional text>]
%%% wrap with braces list environments
\NewDocumentCommand\mybrace{mmo}{%
\IfValueTF {#3}{%
\begin{tikzpicture}[overlay, remember picture,decoration={brace,amplitude=1ex}]
	ack] (#1.north east) -- (#2.south east) node[midway, right=0.1cm]
	{$\Rightarrow$}node[midway, right=0.5cm,text=black,text width = 2in,] {{#3}};
	\end{tikzpicture}%
}%
{%
	\begin{tikzpicture}[overlay, remember
		picture,decoration={brace,amplitude=1ex}]
		ack] (#1.north east) -- (#2.south east);
		\end{tikzpicture}%
	}%
}%a
%%%%%%%How to use this %%%%%%%%%%%%%%%%%%%%%%%%%
%use \tikzmark{a} and \tikzmark{b} at first and last \item where the brace is
%wanted
%use the following command after \end{enumerate}
%\mybrace{a}{b}[Text comes here to describe these to items and justify for your
%case]]


\setcounter{secnumdepth}{2}

 %chapter format
\titleformat{\chapter}[display]%
{\bfseries\itshape\Huge\color{blue!55}}% format applied to label+text
    {\begin{tikzpicture}
            \node[rectangle,fill=blue!55,text=white]{\itshape\fontsize{40}{50}\selectfont
            \thechapter} ;
\end{tikzpicture}}%label
  {1.5\baselineskip}% horizontal separation between label and title body
  {}% before the title body
  []% after the title body

% section format
  \titleformat{\section}%
  {\normalfont\Large\bfseries\itshape\color{blue!55}}% format applied to label+text
  {\llap{\colorbox{blue!55}{\parbox{0.9cm}{\hfill\color{white}\thesection}}}}% label
  {1em}% horizontal separation between label and title body
  {}% before the title body
  [] % after the title body 
% subsection format
\titleformat{\subsection}%
  {\normalfont\large\bfseries\itshape\color{blue!55}}% format applied to label+text
  {\llap{\colorbox{blue!55}{\parbox{0.9cm}{\hfill\color{white}\thesubsection}}}}% label
  {1em}% horizontal separation between label and title body
  {}% before the title body
  []% after the title body 


%%%%%%%%%%%%%%%% Fancy Tables %%%%%%%%%%%%%%%%%%%%%%%%%%%%
\usepackage{calc}
\usepackage{array}
\definecolor{TabLine}{RGB}{254,254,254}
\newcommand{\TabRowHead}{\rowcolor{TabHeadRow}}
\newcommand{\TabRowHeadCor}{\cellcolor{white}}
\newcommand{\TabRowHCol}{\color{white}\bfseries\boldmath}
\newcommand{\TabCellHead}{\cellcolor{TabHeadRow}\TabRowHCol}
\newenvironment{Mytable}%
    {\begingroup\setlength{\arrayrulewidth}{2pt}\arrayrulecolor{TabLine}
    \colorlet{TabHeadRow}{Col\thechapter}
    \colorlet{TabRowOdd}{Col\thechapter!50!white}
    \colorlet{TabRowEven}{Col\thechapter!25!white}
    \rowcolors{1}{TabRowOdd}{TabRowEven}
    }%
    {\endgroup

}

\newcommand{\twocolumnside}[2]{\begin{minipage}[t]{0.45\linewidth}\raggedright
#1
\end{minipage}\hfill{\color{Col\thechapter}{\vrule width 1pt}}\hfill\begin{minipage}[t]{0.45\linewidth}\raggedright
#2
\end{minipage}
}

\newcommand{\twocolumnsides}[2]{\begin{minipage}[t]{0.45\linewidth}\raggedright
#1
\end{minipage}\hfill\begin{minipage}[t]{0.45\linewidth}\raggedright
#2
\end{minipage}
}

